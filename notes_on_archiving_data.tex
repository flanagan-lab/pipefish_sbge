% Options for packages loaded elsewhere
\PassOptionsToPackage{unicode}{hyperref}
\PassOptionsToPackage{hyphens}{url}
%
\documentclass[
]{article}
\title{Archiving data with NCBI and GA}
\author{Emily Beasley}
\date{2025-04-23}

\usepackage{amsmath,amssymb}
\usepackage{lmodern}
\usepackage{iftex}
\ifPDFTeX
  \usepackage[T1]{fontenc}
  \usepackage[utf8]{inputenc}
  \usepackage{textcomp} % provide euro and other symbols
\else % if luatex or xetex
  \usepackage{unicode-math}
  \defaultfontfeatures{Scale=MatchLowercase}
  \defaultfontfeatures[\rmfamily]{Ligatures=TeX,Scale=1}
\fi
% Use upquote if available, for straight quotes in verbatim environments
\IfFileExists{upquote.sty}{\usepackage{upquote}}{}
\IfFileExists{microtype.sty}{% use microtype if available
  \usepackage[]{microtype}
  \UseMicrotypeSet[protrusion]{basicmath} % disable protrusion for tt fonts
}{}
\makeatletter
\@ifundefined{KOMAClassName}{% if non-KOMA class
  \IfFileExists{parskip.sty}{%
    \usepackage{parskip}
  }{% else
    \setlength{\parindent}{0pt}
    \setlength{\parskip}{6pt plus 2pt minus 1pt}}
}{% if KOMA class
  \KOMAoptions{parskip=half}}
\makeatother
\usepackage{xcolor}
\IfFileExists{xurl.sty}{\usepackage{xurl}}{} % add URL line breaks if available
\IfFileExists{bookmark.sty}{\usepackage{bookmark}}{\usepackage{hyperref}}
\hypersetup{
  pdftitle={Archiving data with NCBI and GA},
  pdfauthor={Emily Beasley},
  hidelinks,
  pdfcreator={LaTeX via pandoc}}
\urlstyle{same} % disable monospaced font for URLs
\usepackage[margin=1in]{geometry}
\usepackage{color}
\usepackage{fancyvrb}
\newcommand{\VerbBar}{|}
\newcommand{\VERB}{\Verb[commandchars=\\\{\}]}
\DefineVerbatimEnvironment{Highlighting}{Verbatim}{commandchars=\\\{\}}
% Add ',fontsize=\small' for more characters per line
\usepackage{framed}
\definecolor{shadecolor}{RGB}{248,248,248}
\newenvironment{Shaded}{\begin{snugshade}}{\end{snugshade}}
\newcommand{\AlertTok}[1]{\textcolor[rgb]{0.94,0.16,0.16}{#1}}
\newcommand{\AnnotationTok}[1]{\textcolor[rgb]{0.56,0.35,0.01}{\textbf{\textit{#1}}}}
\newcommand{\AttributeTok}[1]{\textcolor[rgb]{0.77,0.63,0.00}{#1}}
\newcommand{\BaseNTok}[1]{\textcolor[rgb]{0.00,0.00,0.81}{#1}}
\newcommand{\BuiltInTok}[1]{#1}
\newcommand{\CharTok}[1]{\textcolor[rgb]{0.31,0.60,0.02}{#1}}
\newcommand{\CommentTok}[1]{\textcolor[rgb]{0.56,0.35,0.01}{\textit{#1}}}
\newcommand{\CommentVarTok}[1]{\textcolor[rgb]{0.56,0.35,0.01}{\textbf{\textit{#1}}}}
\newcommand{\ConstantTok}[1]{\textcolor[rgb]{0.00,0.00,0.00}{#1}}
\newcommand{\ControlFlowTok}[1]{\textcolor[rgb]{0.13,0.29,0.53}{\textbf{#1}}}
\newcommand{\DataTypeTok}[1]{\textcolor[rgb]{0.13,0.29,0.53}{#1}}
\newcommand{\DecValTok}[1]{\textcolor[rgb]{0.00,0.00,0.81}{#1}}
\newcommand{\DocumentationTok}[1]{\textcolor[rgb]{0.56,0.35,0.01}{\textbf{\textit{#1}}}}
\newcommand{\ErrorTok}[1]{\textcolor[rgb]{0.64,0.00,0.00}{\textbf{#1}}}
\newcommand{\ExtensionTok}[1]{#1}
\newcommand{\FloatTok}[1]{\textcolor[rgb]{0.00,0.00,0.81}{#1}}
\newcommand{\FunctionTok}[1]{\textcolor[rgb]{0.00,0.00,0.00}{#1}}
\newcommand{\ImportTok}[1]{#1}
\newcommand{\InformationTok}[1]{\textcolor[rgb]{0.56,0.35,0.01}{\textbf{\textit{#1}}}}
\newcommand{\KeywordTok}[1]{\textcolor[rgb]{0.13,0.29,0.53}{\textbf{#1}}}
\newcommand{\NormalTok}[1]{#1}
\newcommand{\OperatorTok}[1]{\textcolor[rgb]{0.81,0.36,0.00}{\textbf{#1}}}
\newcommand{\OtherTok}[1]{\textcolor[rgb]{0.56,0.35,0.01}{#1}}
\newcommand{\PreprocessorTok}[1]{\textcolor[rgb]{0.56,0.35,0.01}{\textit{#1}}}
\newcommand{\RegionMarkerTok}[1]{#1}
\newcommand{\SpecialCharTok}[1]{\textcolor[rgb]{0.00,0.00,0.00}{#1}}
\newcommand{\SpecialStringTok}[1]{\textcolor[rgb]{0.31,0.60,0.02}{#1}}
\newcommand{\StringTok}[1]{\textcolor[rgb]{0.31,0.60,0.02}{#1}}
\newcommand{\VariableTok}[1]{\textcolor[rgb]{0.00,0.00,0.00}{#1}}
\newcommand{\VerbatimStringTok}[1]{\textcolor[rgb]{0.31,0.60,0.02}{#1}}
\newcommand{\WarningTok}[1]{\textcolor[rgb]{0.56,0.35,0.01}{\textbf{\textit{#1}}}}
\usepackage{graphicx}
\makeatletter
\def\maxwidth{\ifdim\Gin@nat@width>\linewidth\linewidth\else\Gin@nat@width\fi}
\def\maxheight{\ifdim\Gin@nat@height>\textheight\textheight\else\Gin@nat@height\fi}
\makeatother
% Scale images if necessary, so that they will not overflow the page
% margins by default, and it is still possible to overwrite the defaults
% using explicit options in \includegraphics[width, height, ...]{}
\setkeys{Gin}{width=\maxwidth,height=\maxheight,keepaspectratio}
% Set default figure placement to htbp
\makeatletter
\def\fps@figure{htbp}
\makeatother
\setlength{\emergencystretch}{3em} % prevent overfull lines
\providecommand{\tightlist}{%
  \setlength{\itemsep}{0pt}\setlength{\parskip}{0pt}}
\setcounter{secnumdepth}{-\maxdimen} % remove section numbering
\ifLuaTeX
  \usepackage{selnolig}  % disable illegal ligatures
\fi

\begin{document}
\maketitle

{
\setcounter{tocdepth}{2}
\tableofcontents
}
There are a number of public repositories in which you may wish to
archive your data. This RMarkdown follows the steps I took to prepare my
metadata and upload raw reads to NCBI Sequence Read Archive (SRA) and
Genomics Aotearoa, as well as outlining best practices for choosing an
appropriate repository for RNAseq reads. While I am uploading RNAseq raw
reads, much of this information is broadly applicable to other sample
types.

\hypertarget{why-archive-data-in-public-repositories}{%
\section{Why archive data in public
repositories?}\label{why-archive-data-in-public-repositories}}

Archiving raw reads in a public repository like NCBI SRA is crucial for
multiple reasons - in terms of scientific integrity, in accordance with
principles of kaitiakitanga and data sovereignty (for samples collected
in Aotearoa New Zealand), and in alignment with \emph{FAIR} and
\emph{CARE} principles. Here's a brief breakdown of some of the key
reasons to archive data:

\hypertarget{why-archive-raw-reads-at-all}{%
\subsection{Why Archive Raw Reads at
All?}\label{why-archive-raw-reads-at-all}}

\begin{enumerate}
\def\labelenumi{\arabic{enumi}.}
\item
  Reproducibility: Raw data allows other researchers to reproduce
  results, validate findings, and apply new analysis techniques as they
  emerge.
\item
  Transparency: It promotes transparency in the scientific
  process---others can see exactly what data were used and how.
\item
  Data longevity: Journals or institutions might disappear or change
  policy, but established repositories ensure long-term access.
\item
  Reuse and meta-analysis: Raw reads can be re-analysed for different
  research questions, aggregated in meta-analyses, or used to benchmark
  tools.
\item
  Citations and credit: Publicly archived datasets are citable,
  providing credit to the data generators, and increasing impact.
\end{enumerate}

\hypertarget{fair-principles-wilkinson-et-al.-2016}{%
\subsection{FAIR Principles (Wilkinson et al.,
2016)}\label{fair-principles-wilkinson-et-al.-2016}}

\emph{Findable, Accessible, Interoperable, Reusable}

\begin{enumerate}
\def\labelenumi{\arabic{enumi}.}
\item
  \emph{Findability}: Data is indexed, assigned DOIs or accession
  numbers, and searchable.
\item
  \emph{Accessibility}: The data is publicly available with standard
  access protocols.
\item
  \emph{Interoperability}: Standardised formats (e.g., FASTQ, BAM) and
  metadata schemas make data usable across platforms.
\item
  \emph{Reusability}: Clear licensing and metadata help others
  understand and repurpose the data appropriately.
\end{enumerate}

\hypertarget{care-principles-carroll-et-al.-2020}{%
\subsection{CARE Principles (Carroll et al.,
2020)}\label{care-principles-carroll-et-al.-2020}}

\emph{Collective Benefit, Authority to Control, Responsibility, Ethics}

CARE applies especially to Indigenous data and culturally sensitive
genomic data, adding a layer of ethical responsibility:

\begin{enumerate}
\def\labelenumi{\arabic{enumi}.}
\item
  \emph{Collective Benefit}: Ensures the data benefits the communities
  it originates from --- not just external researchers.
\item
  \emph{Authority to Control}: Recognises the rights of communities to
  govern their data.
\item
  \emph{Responsibility}: Encourages researchers to steward data
  responsibly, ensuring it's used ethically.
\item
  \emph{Ethics}: Respects cultural norms and consent, especially when
  working with human or community-associated samples.
\end{enumerate}

\hypertarget{cases-in-which-you-should-upload-your-data-to-genomics-aotearoa}{%
\section{Cases in which you should upload your data to Genomics
Aotearoa}\label{cases-in-which-you-should-upload-your-data-to-genomics-aotearoa}}

\begin{enumerate}
\def\labelenumi{\arabic{enumi}.}
\item
  Your project is funded or affiliated with Genomics Aotearoa --
  Uploading data is part of their data-sharing policies and contributes
  to national efforts in genomics.
\item
  You are ready to share your genomic data for storage, access, or
  reuse---either for publication, collaboration, or long-term archiving.
\item
  You are preparing to publish your research, and data deposition is
  required for transparency, reproducibility, or to comply with journal
  or funding requirements.
\item
  You've completed appropriate quality checks and metadata
  documentation, ensuring your data meet FAIR principles.
\item
  You have ethical and Māori data governance considerations in place,
  especially if your data involves taonga species or Māori participants,
  as GA promotes a Te Ao Māori framework and Indigenous data
  sovereignty.
\item
  You're transitioning from active analysis to archival, and you want to
  ensure long-term storage in a New Zealand-based, trusted genomics
  repository.
\end{enumerate}

\hypertarget{preparing-and-uploading-data-to-ncbi}{%
\section{Preparing and uploading data to
NCBI}\label{preparing-and-uploading-data-to-ncbi}}

To upload data to the NCBI Sequence Read Archive (SRA), you'll need to
go through several steps involving account creation, metadata
preparation, data formatting, and actual submission. Submission through
the portal on the \href{https://www.ncbi.nlm.nih.gov/}{NCBI website} is
fairly straightforward, but I'll go through each step:

\hypertarget{create-an-ncbi-account}{%
\subsection{Create an NCBI account}\label{create-an-ncbi-account}}

Go to \href{https://www.ncbi.nlm.nih.gov/}{NCBI website} and create an
account if you don't already have one. You can link your ORCHID with
your NCBI account and use the same log in information if you already
have an ORCHID.

\hypertarget{register-a-bioproject-and-biosample}{%
\subsection{Register a BioProject and
BioSample}\label{register-a-bioproject-and-biosample}}

These act as containers for your sequencing data and metadata:

\emph{BioProject}

\begin{enumerate}
\def\labelenumi{\arabic{enumi}.}
\item
  Go to the BioProject Submission Portal
  \href{https://submit.ncbi.nlm.nih.gov/subs/bioproject/}{BioProject}
\item
  Provide information about your project (title, description,
  objectives, etc.)
\item
  Submit and save the BioProject accession number (e.g., PRJNA123456)
\end{enumerate}

\emph{BioSample}

\begin{enumerate}
\def\labelenumi{\arabic{enumi}.}
\item
  Go to the BioSample Submission Portal
  \href{https://submit.ncbi.nlm.nih.gov/subs/biosample/}{BioSample}
\item
  Link each sample to your BioProject
\item
  Describe each sample (organism, tissue, treatment, etc.)
\item
  Submit and save the BioSample accession numbers (e.g., SAMN12345678)
\end{enumerate}

\hypertarget{prepare-your-data}{%
\subsection{Prepare your data}\label{prepare-your-data}}

\begin{enumerate}
\def\labelenumi{\arabic{enumi}.}
\item
  Use a standard format - my raw reads are all \emph{FASTQ} (compressed
  as \texttt{.fastq.gz})
\item
  Ensure proper naming conventions (e.g., \texttt{sample1\_R1.fastq.gz})
\end{enumerate}

\hypertarget{create-an-sra-submission}{%
\subsection{Create an SRA submission}\label{create-an-sra-submission}}

\begin{enumerate}
\def\labelenumi{\arabic{enumi}.}
\item
  Go to the \href{https://submit.ncbi.nlm.nih.gov/subs/sra/}{SRA
  Submission Portal}
\item
  Select the BioProject you registered earlier
\item
  Link to your BioSample(s)
\item
  Fill out the library preparation details (platform, layout, library
  strategy, etc.)
\item
  Fill out the metadata spreadsheet (SRA provides one in the portal - it
  is recommended to fill out the spreadsheet in the portal)
\end{enumerate}

\hypertarget{upload-your-files}{%
\subsection{Upload your files}\label{upload-your-files}}

There are three main methods

\begin{enumerate}
\def\labelenumi{\arabic{enumi}.}
\item
  Web upload
\item
  Aspera Upload
\item
  FTP upload
\end{enumerate}

I uploaded my files with FTP in the University of Canterbury's Remote
Computing Cluster (RCC). In order to do this, follow these steps:

\begin{enumerate}
\def\labelenumi{\arabic{enumi}.}
\item
  When you get to step 7 \emph{Files}, select ``FTP upload'' as your
  transfer method. This will create a request for a directory into which
  you can upload your files.
\item
  From the SRA Submission Portal, you can obtain:

  \begin{itemize}
  \item
    Server address: ftp-private.ncbi.nlm.nih.gov
  \item
    Username: subftp
  \item
    Password: {[}Provided in the portal{]}
  \item
    Account folder path: uploads/{[}your.email\_xxxxx{]}
  \end{itemize}
\item
  Upload files through the RCC

  \begin{itemize}
  \item
    Navigate into the directory which contains \emph{only the reads you
    wish to upload}.
  \item
    Start a screen session using the command \texttt{screen}. This is
    essential, as an upload may take several days. By running your ftp
    in a screen session, it will not time out or disconnect.
  \end{itemize}
\end{enumerate}

\begin{Shaded}
\begin{Highlighting}[]
\CommentTok{\# Initialise FTP connection}
\FunctionTok{ftp} \AttributeTok{{-}i}

\CommentTok{\#Establish FTP connection}
\FunctionTok{ftp}\NormalTok{ ftp{-}private.ncbi.nlm.nih.gov}

\CommentTok{\# Login with credentials }
\ExtensionTok{User:}\NormalTok{ subftp}
\ExtensionTok{Password:}\NormalTok{ [provided password]}

\CommentTok{\# Navigate to your account folder}
\BuiltInTok{cd}\NormalTok{ uploads/your.email\_xxxxx}

\CommentTok{\# Create a submission folder}
\FunctionTok{mkdir}\NormalTok{ your\_submission\_folder}

\CommentTok{\# Change to the submission directory}
\BuiltInTok{cd}\NormalTok{ your\_submission\_folder}

\CommentTok{\# Upload your files}
\ExtensionTok{mput} \PreprocessorTok{*}\NormalTok{.fastq.gz}
\end{Highlighting}
\end{Shaded}

The prompt \texttt{{[}anpqy?{]}?} appears during FTP operations when
using commands like mget (multiple get) or mput (multiple put). It's
asking how you want to handle multiple files in the operation.

Each letter represents a different option:

a - Accept all remaining files without further prompts

n - Skip all remaining files

p - Toggle prompt mode (turn it off/on)

q - Quit the current operation

y - Accept this specific file

? - Display help about these options

Common scenarios include:

Type `y' to accept each file individually

Type `a' to accept all remaining files without further prompts

Type `n' to skip all remaining files

Type `p' to turn off prompting completely (useful for large transfers)

Choose whichever option is appropriate for you - I chose p as my
transfer was large.

Once this is running, \emph{detach from the screen session} by using
\texttt{ctrl+a\ d}. If you do not detach, it may interrupt the
connection when you shut your computer down or exit the RCC.

You can check on the progress of your file submission in two ways:

\begin{enumerate}
\def\labelenumi{\arabic{enumi}.}
\item
  You can resume your screen session using the command:
  \texttt{screen\ -r\ {[}pid.{]}tty.host}. Replace
  \texttt{{[}pid.{]}tty.host} with your chosen screen - it will look
  something like: \texttt{1234567.pts-1.UCRCC0494}. Remember to detach
  once you're done.
\item
  You can return to the SRA Submission Portal, click on the \emph{Select
  preloaded folder}, and view the number of files uploaded. There is a
  lag of about 10 minutes on the SRA Submission Portal page, so your
  best bet is to reattach your screen.
\end{enumerate}

\hypertarget{finalise-and-submit}{%
\subsection{Finalise and submit}\label{finalise-and-submit}}

\begin{enumerate}
\def\labelenumi{\arabic{enumi}.}
\item
  Review your metadata and file associations
\item
  Click \emph{Submit}
\item
  You'll receive confirmation with your SRA accession numbers (e.g.,
  SRR12345678)
\end{enumerate}

\hypertarget{after-submission}{%
\subsection{After submission}\label{after-submission}}

\begin{enumerate}
\def\labelenumi{\arabic{enumi}.}
\item
  Your data will be reviewed and validated
\item
  You'll get an email once your data is live or if any errors need
  fixing
\item
  You can choose to release the data immediately or hold until
  publication. You can change the release data at any time.
\end{enumerate}

\hypertarget{preparing-and-uploading-data-to-aotearoa-genomic-data-repository-agdr}{%
\section{Preparing and uploading data to Aotearoa Genomic Data
Repository
(AGDR)}\label{preparing-and-uploading-data-to-aotearoa-genomic-data-repository-agdr}}

To upload data to the Aotearoa Genomic Data Repository (AGDR), you'll
need to go through several steps involving account creation, metadata
preparation, data formatting, and actual submission. Submission through
AGDR platform is not as straightforward as NCBI, but there is a lot of
guidance on the \href{https://data.agdr.org.nz/}{AGDR website}. While
many of the requirements are similar to NCBI, there are some important
differences, so I'll go through each step:

\hypertarget{access-the-agdr-platform}{%
\subsection{Access the AGDR platform}\label{access-the-agdr-platform}}

\begin{enumerate}
\def\labelenumi{\arabic{enumi}.}
\item
  Visit: \href{https://data.agdr.org.nz/}{AGDR website}
\item
  Create an account and log in
\end{enumerate}

\hypertarget{understand-submission-requirements}{%
\subsection{Understand submission
requirements}\label{understand-submission-requirements}}

\begin{enumerate}
\def\labelenumi{\arabic{enumi}.}
\item
  Data Types: AGDR accepts non-human genomic data, particularly from
  biological and environmental samples originating in Aotearoa New
  Zealand.
\item
  Metadata: Prepare detailed metadata for your dataset, including
  information about the species, sample collection, sequencing methods,
  and any associated publications
\end{enumerate}

\hypertarget{suitability-assessment}{%
\subsection{Suitability assessment}\label{suitability-assessment}}

\begin{enumerate}
\def\labelenumi{\arabic{enumi}.}
\item
  Fill out the application form found here:
  \href{https://docs.agdr.org.nz/user_guides/submission_process/}{ADGR
  Submission Process}
\item
  After review of your application, you will need to fill out the
  metadata template (see below) and you will receive specific
  instructions on how to transfer your data via Globus.
\end{enumerate}

\hypertarget{prepare-your-data-1}{%
\subsection{Prepare your data}\label{prepare-your-data-1}}

\begin{enumerate}
\def\labelenumi{\arabic{enumi}.}
\item
  Data formats: ensure your data files are in accepted formats (e.g.,
  FASTQ, BAM, VCF)
\item
  Metadata template: use the provided templates to structure your
  metadata appropriately. The template can be found here:
  \href{https://docs.google.com/spreadsheets/d/1be8T4JalxRopM7pVFzTuempBGlqqhMQ_8MryGsjHI5E/edit?gid=188280636\#gid=188280636}{ADGR
  template}. Download it and fill out the required information. I have
  written two bash scripts (below) to help fill in some of the metadata
  required by ADGR.
\end{enumerate}

Firstly, ADGR requires the md5sum for each file you're submitting.

Make sure you run this script in the directory from where you will be
depositing your files. I am depositing my files from the FlanaganLab
research folder, which I mounted in the RCC. Instructions on how to
mount this folder can be found here:
\href{https://wiki.canterbury.ac.nz/display/RCC/Data+transfer}{SMB
mount}.

\begin{Shaded}
\begin{Highlighting}[]
\CommentTok{\#!/bin/bash}

\CommentTok{\# Loop through all files in the current directory}
\ControlFlowTok{for}\NormalTok{ file }\KeywordTok{in} \PreprocessorTok{*}\KeywordTok{;} \ControlFlowTok{do}
    \CommentTok{\# Only process regular files (skip directories)}
    \ControlFlowTok{if} \BuiltInTok{[} \OtherTok{{-}f} \StringTok{"}\VariableTok{$file}\StringTok{"} \BuiltInTok{]}\KeywordTok{;} \ControlFlowTok{then}
        \CommentTok{\# Calculate and display MD5 sum}
        \ExtensionTok{md5sum} \StringTok{"}\VariableTok{$file}\StringTok{"}
    \ControlFlowTok{fi}
\ControlFlowTok{done}
\end{Highlighting}
\end{Shaded}

This script was run as
\texttt{bash\ ../../../../../home/rccuser/shared/emily\_files/scripts/md5sum.sh\ \textgreater{}\ checksums.txt}

Secondly, ADGR requires file sizes in bytes for each of the files you
are submitting.

\begin{Shaded}
\begin{Highlighting}[]
\ExtensionTok{!/bin/bash}
\CommentTok{\#Create the arguments}
\VariableTok{input\_dir}\OperatorTok{=}\VariableTok{$1}

\CommentTok{\# Calculate size for each file}
\ControlFlowTok{for}\NormalTok{ file }\KeywordTok{in} \StringTok{"}\VariableTok{$1}\StringTok{"}\NormalTok{/}\PreprocessorTok{*}\KeywordTok{;} \ControlFlowTok{do}
    \ControlFlowTok{if} \BuiltInTok{[} \OtherTok{{-}f} \StringTok{"}\VariableTok{$file}\StringTok{"} \BuiltInTok{]}\KeywordTok{;} \ControlFlowTok{then}
        \CommentTok{\# Use stat for GNU systems (Linux)}
        \VariableTok{size}\OperatorTok{=}\VariableTok{$(}\FunctionTok{stat} \AttributeTok{{-}c\%s} \StringTok{"}\VariableTok{$file}\StringTok{"} \DecValTok{2}\OperatorTok{\textgreater{}}\NormalTok{/dev/null}\VariableTok{)}
        \BuiltInTok{echo} \StringTok{"}\VariableTok{$\{file}\OperatorTok{\#\#}\PreprocessorTok{*}\NormalTok{/}\VariableTok{\}}\StringTok{: }\VariableTok{$size}\StringTok{ bytes"}
    \ControlFlowTok{fi}
\ControlFlowTok{done}
\end{Highlighting}
\end{Shaded}

This script was run as
\texttt{bash\ ../../../../../home/rccuser/shared/emily\_files/scripts/get\_file\_size.sh\ \textgreater{}\ filesize.txt}

\hypertarget{submit-your-data}{%
\subsection{Submit your data}\label{submit-your-data}}

\begin{enumerate}
\def\labelenumi{\arabic{enumi}.}
\item
  Upload process: Log in to your AGDR account and follow the submission
  process to upload your data and associated metadata through Globus.
\item
  Create a Globus account using your institutional log in here
  \href{https://www.globus.org/}{Globus}
\item
  ADGR doesn't typically accept FTP uploads, so you'll have to download
  and install Globus Connect Personal:
  \href{https://www.globus.org/globus-connect-personal}{Globus Connect}
\item
  Set it up to make your local machine a personal endpoint
\item
  ADGR will provide the name of the Globus endpoint you should send your
  files to (e.g., ADGR\#ProjectName)
\item
  Log in to the Globus Web App \href{https://app.globus.org/}{Globus Web
  App}
\item
  Log in and open the \emph{File Manager}
\item
  In the File Manager:

  \begin{enumerate}
  \def\labelenumii{\alph{enumii})}
  \item
    On the left pane, select your local endpoint (your computer or
    institution).
  \item
    On the right pane, search and select the ADGR endpoint.
  \item
    Navigate to the correct folders on both sides
  \item
    Select the files you want to transfer and click \emph{Start}
  \end{enumerate}
\item
  Review: Your submission will be reviewed for completeness and
  compliance with AGDR guidelines.
\end{enumerate}

\hypertarget{data-access-and-governance}{%
\subsection{Data access and
governance}\label{data-access-and-governance}}

\begin{enumerate}
\def\labelenumi{\arabic{enumi}.}
\item
  Māori data sovereignty: AGDR operates under principles that respect
  Māori data sovereignty. If your data pertains to taonga species, it
  will be subject to kaitiaki (guardian) approval for access and use.
\item
  Access requests: Researchers can request access to datasets, but
  approval is granted by the designated kaitiaki. This process ensures
  that data use aligns with Māori values and expectations.
\end{enumerate}

\hypertarget{additional-resources}{%
\subsection{Additional resources}\label{additional-resources}}

\begin{enumerate}
\def\labelenumi{\arabic{enumi}.}
\item
  ADGR documentation: for detailed guidance, visit the
  \href{https://docs.agdr.org.nz/}{ADGR User Guide}
\item
  Māori ethical frameworks: Familiarise yourself with guidelines like
  \href{https://www.genomics-aotearoa.org.nz/our-work/completed-projects/te-nohonga-kaitiaki}{Te
  Nohonga Kaitiaki}
\end{enumerate}

\end{document}
